\section{Curse of dimensionality}

\textit{In Progress.}

Show figures of three triangles. Students must use the vertices to compute side lengths, interior angles, and areas.  TODO: Find a simple measure of the `quality' of a triangle, (i.e. how close it is to equilateral)
The project is to determine whether the expected quality of random triangles in R2  is different than random triangles in R10. 
\begin{enumerate}
  \item Use a normal random number generator to create three random points in R2.
  \item Let these points be the vertices of a triangle. Compute the lengths and interior angles of the triangle.
  \item Repeat, and histogram triangle quality scores
\end{enumerate}

Repeat process for triangls in R10, (i.e. use a multivariate normal to generate 3 points in R10 and analze the triange.
Can you find an heuristic, geometric explanation for why the triangles in high dimensions are much closer to equilateral?
