\section{Eigenvectors of Circulant Matrices}


Define the $16 \times 16$ matrix $A$ by
\begin{equation}
A = \begin{bmatrix*}[r] -2 & 1 & 0 & 0 & \dots & 0 & 0 & 1 \\ 1 & -2 & 1 & 0 & \dots& 0 & 0 & 0 \\ 0 & 1 & -2 & 1 & \cdots&0 & 0 & 0 \\ 0 & 0 & 1 & -2 & \cdots&0 & 0 & 0 \\ \vdots & \vdots & \vdots & \vdots & \ddots & \vdots & \vdots & \vdots \\ 0 & 0 & 0 & 0 & \cdots & -2 & 1 & 0 \\ 0 & 0 & 0 & 0 & \cdots & 1 & -2 & 1\\ 1 & 0 & 0 & 0 & \cdots & 0 & 1 & -2 \end{bmatrix*}
\end{equation}
In this project, you will explore (and explain) the behavior of the iterative map
    \begin{equation}
      \label{eq:eigen-iterative}
      \bm{x}_{k+1} = \left(I + \eta A\right) \bm{x}_{k}
    \end{equation}
   for various initial vectors $\bm{x}_0$, and for $\eta = 0.1$. 
\begin{enumerate}[(a)]
  \item Create a variable for the matrix $A$ by using the function \texttt{A=diag(...)} (Matlab), or \texttt{A=np.diag(...)} (Python), or \texttt{diagm(...)} (Julia).
  \item Try different initial vectors $\bm{x}_0$ and plot $\bm{x}_1$, $\bm{x}_2, \dots $. Can you qualitatively describe the behavior of the matrix $(I + \eta A)$? 
  \item Compute the eigenvectors of $(I+\eta A)$ by using the function \texttt{[X,V]=eig(...)} (Matlab), \texttt{V,X = np.linalg.eig(...)} (Python), or \texttt{V,X=eigen(...)} (Julia).
  \item \sloppy Plot enough of the eigenvectors so you get a sense for what they look like.\\ \textit{Hint:} If your software returned complex-valued eigenvectors, it may be helpful to plot real and imaginary components separately (e.g. in Julia, this would be \texttt{plot(real(X[k,:]))} and \texttt{plot(imag(X[k,:]))}). 
  \item Propose a closed form representation for each of the eigenvectors.
  \item[($\ast$)] \textit{Bonus:} Can you prove that every symmetric circulant matrix has eigenvectors in the form your found in part (e).
  \item Without any further computation, and by only considering eigenvectors and their corresponding eigenvalues, describe the behavior of equation (\ref{eq:eigen-iterative}) for an initial vector in the form
\begin{equation*}
\bm{x}_0 = \big(\, \cos(2\pi k / 16) + 5 \cos( 4 \cdot 2\pi k / 16), \text{ for } k = 0,...,15\,\big).
\end{equation*}
\end{enumerate}
Repeat these steps for the $16 \times 16$ matrix $B$, defined by
\begin{equation}
B = \begin{bmatrix*}[r] 0 & 1 & 0 & 0 & \dots & 0 & 0 & -1 \\ -1 & 0 & 1 & 0 & \dots& 0 & 0 & 0 \\ 0 & -1 & 0 & \phantom{+}1 & \cdots&0 & 0 & 0 \\ 0 & 0 & -1 & 0 & \cdots&0 & 0 & 0 \\ \vdots & \vdots & \vdots & \vdots & \ddots & \vdots & \vdots & \vdots \\ 0 & 0 & 0 & 0 & \cdots & 0 & 1 & 0 \\ 0 & 0 & 0 & 0 & \cdots & -1 & 0 & 1\\ 1 & 0 & 0 & 0 & \cdots & 0 & -1 & 0 \end{bmatrix*}
\end{equation}
