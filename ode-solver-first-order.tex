% Project for Integration Workshop. Dept. Mathematics. UArizona
% Write a first-order ODE solver for a system of ODEs

\section{First-order ODE Integrators}
Consider the differential equation
\begin{equation}
\label{eq:ode_system}
\begin{cases} \dot{x}(t) = -y, & \quad x(0) = 1\\ \dot{y}(t) = \phantom{-}x, & \quad y(0) = 0 \end{cases}
\end{equation}
% which we can also write in matrix notation as
% \begin{equation}
% \label{eq:ode_matrix}
% \dot{\bm{x}}(t) = \begin{bmatrix*}[r] 0 & -1 \\ 1 & 0 \end{bmatrix*} \bm{x}, \quad \bm{x}(0) = \begin{pmatrix} 1\\0 \end{pmatrix}
% \end{equation}
Use pencil-and-paper to solve the ODE. Make a quiver plot. Sketch solutions.\\

In this project, you will build a numerical solver for differential equations like this one. You will test your solver on this ODE, and then use it to solve a more interesting ODE.
For this problem, discretize the time interval $0 \leq t \leq T$ into $n+1$ points. That is, set $\Delta t = T/n$ and $t_k = k \, \Delta t$, so $ t = \{t_0, t_1, \dots, t_n\}$. \\
\textit{Forward Differences:}
\begin{enumerate}[(a)]
    \item Use what's called a \textit{first-order forward difference} approximation to the derivatives $\dot{x}$ and $\dot{y}$ at each $t$ as follows:
    \begin{equation} 
    \label{eq:forward_difference}
    \dot{x}(t) \approx \frac{x(t + \Delta t) - x(t)}{\Delta t} \qquad \text{and} \qquad \dot{y}(t) \approx \frac{y(t + \Delta t) - y(t)}{\Delta t}.
    \end{equation}
     Substitute equation (\ref{eq:forward_difference}) into equation (\ref{eq:ode_system}) and show that we can approximate the differential equation by the discrete time-stepping process:
    \begin{equation}
    \label{eq:forward_euler}
    \begin{cases} x_{k+1} = x_k - (\Delta t) y_k, & \quad x_0 = 1\\  y_{k+1} = y_k + (\Delta t) x_k, & \quad y_0 = 0\\  \end{cases}
    \end{equation}
    \item Implement equation \ref{eq:forward_euler} for $0 \leq t \leq 1$ and plot the trajectory given by the solution.
    \item Is your approximation qualitatively correct? Will it remain qualitatively correct on $0 \leq t \leq T$ as $T$ gets very large? Explain.\\
    \textit{Hint:} It may be useful to rewrite the system of difference equations in matrix form,
    \begin{equation*}
    \bm{x}_{k+1} 
    % = \bm{x}_n + (\Delta t) \begin{bmatrix*}[r] 0 & -1 \\ 1 & 0 \end{bmatrix*} \bm{x}_n 
    = \begin{bmatrix} 1 & -\Delta t \\  \Delta t & 1\end{bmatrix} \bm{x}_k, \quad \bm{x}_0 = \begin{pmatrix} 1\\0 \end{pmatrix}
    \end{equation*}
    and to analyze the eigenvalues of the matrix.
\end{enumerate}
\textit{Backward Differences:}
\begin{enumerate}[(a),resume]
    \item We could have solved this problem by a different approach. We could also have used what's called a \textit{first-order backward difference}:
    \begin{equation} 
    \label{eq:backward_difference}
    \dot{x}(t) \approx \frac{x(t) - x(t - \Delta t)}{\Delta t} \qquad \text{and} \qquad \dot{y}(t) \approx \frac{y(t) - y(t  \Delta t)}{\Delta t}.
    \end{equation}
    Show that the backward difference gives the approximation
    \begin{equation}
    \label{eq:backward_euler}
     \begin{cases} x_{k} = x_{k-1} - (\Delta t) y_k, & \quad x_0 = 1\\  y_{k} = y_{k-1} + (\Delta t) x_k, & \quad y_0 = 0\\  \end{cases}
    \end{equation}
    which can be rewritten as
   \begin{equation*} 
    \bm{x}_{k} 
    % = \bm{x}_n + (\Delta t) \begin{bmatrix*}[r] 0 & -1 \\ 1 & 0 \end{bmatrix*} \bm{x}_n 
    = \left(\begin{bmatrix} 1 & \Delta t \\  -\Delta t & 1\end{bmatrix}\right)^{-1} \bm{x}_{k-1}, \quad \bm{x}_0 = \begin{pmatrix} 1\\0 \end{pmatrix}
    \end{equation*}
    \item Implement equation (\ref{eq:backward_euler}) for $0 \leq t \leq 1$ and plot the trajectory given by the solution. 
    \item Is your approximation qualitatively correct? Will it remain qualitatively correct on $0 \leq t \leq T$ as $T$ gets very large? Explain.
\end{enumerate}
\textit{Bonus:}
\begin{enumerate}[(a),resume]
    \item (The really fun stuff) Use your ODE solver to solve $\dot{x} = -y$, $\dot{y} = \sin x$. 
\end{enumerate}
\newpage

