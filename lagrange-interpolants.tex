% Project for Integration Workshop. Dept. Mathematics. UArizona
% Compute and plot lagrange interpolating polynomials. Explore when they do well and when they do poorly. `Discovery' of chebychev points.

\section{Lagrange Interpolating Polynomials for Function Approximation}
Let $\{(x_k,y_k) , k = 1, \dots, N\}$ be $N$ points in $\mathbb{R}^2$ that satisfy $x_k \neq x_j$ whenever $k \neq j$, the \textit{Lagrange Interpolating polynomial} is defined as
\begin{equation}
L(x) = \sum_{k = 1}^N \left(\prod_{j \neq k}\frac{x-x_j}{x_k-x_j} \right) y_k.
\end{equation}
We wish to determine how well a Lagrange Interpolating polynomial can approximate the functions 
\begin{equation}
f(x) = (x-.9)(x-.4)(x+.1)(x+.7)(x+.8)
\end{equation}
and 
\begin{equation}
g(x) = \frac{1}{1+10x^2} \quad \quad \text{for } -1 \leq x \leq 1
\end{equation}
\begin{enumerate}[(a)]
    \item The Lagrange interpolating polynomial is linear combination of $N$ terms in the form
    \begin{equation}
    T_k(x): =\left( \prod_{j \neq k}\frac{x-x_j}{x_k-x_j}  \right) 
    \end{equation}
    Let $x_1= -1$, $x_2 = 0$, and $x_3 = 1$. Plot the functions $T_k(x)$ for $k = 1,\dots,3$, and describe what you see.
    \item 
    \begin{enumerate}[i.]
      \item Sample $f(x)$ at the points $x_1 = -1$, $x_2 = 0$, and $x_3= 1$. Compute and plot the Lagrange Interpolating polynomial given by these 3 equi-spaced points.
      \item Repeat part (b)(i) with 5 equi-spaced points, and again with 9 equi-spaced points. Plot your results and describe whether your approximation improves with more points. 
    \end{enumerate}
    \item Repeat part (b) for the function $g(x)$.
    \item There is a well-known rule of thumb for improving numerical approximations: Sample your function with a higher density of sample points in regions where the error is biggest. Does this rule of thumb seem to work with the functions $f$ and $g$?
\end{enumerate}

